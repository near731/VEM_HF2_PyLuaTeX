%Preamble
\documentclass[12pt,a4paper]{article}
\usepackage[top=20mm, bottom=20mm, left=20mm, right=20mm]{geometry}
\usepackage[utf8]{inputenc}
\usepackage{fancyhdr} \setlength{\headheight}{15pt}
\usepackage{lastpage}
\usepackage{t1enc}
\usepackage[magyar]{babel}
\usepackage{amsmath}
\usepackage{amssymb,physics}
\usepackage{amsthm}
\usepackage{empheq}
\usepackage{enumitem}
\usepackage{float}
\usepackage{verbatim}
\usepackage{hyperref}
\usepackage{multicol}
\usepackage{blkarray}
\usepackage{mathtools}
\usepackage[most]{tcolorbox}
\usepackage{tikz}
\usepackage{pgfplots}
\usepackage{picture}
\usepackage{graphicx}
\usepackage{makecell}
\usepackage{eso-pic}
\usepackage[executable=python3.10.exe]{pyluatex}

\pagestyle{fancy}
\def\mx#1{\mathbf{#1}}
\def\vec#1{\underline{\mathbf{#1}}}
\def\m{\; \left[\mathrm{m}\right]}
\def\mili{\; \left[\mathrm{mm}\right]}
\def\mm{\; \left[\mathrm{m^2}\right]}
\def\mmmm{\; \left[\mathrm{m^4}\right]}
\def\i{\left(i\right)}
\def\SI{\; \left[\mathrm{SI}\right]}
\def\Nm{\; \left[\mathrm{Nm}\right]}
\def\N{\; \left[\mathrm{N}\right]}
\def\kN{\; \left[\mathrm{kN}\right]}
\def\futyi{\cdot 10^{9}}
\def\fos{\; \left[-\right]}
\def\Gpa{\; \left[\mathrm{GPa}\right]}
\def\rads{\; \left[\mathrm{\frac{rad}{s}}\right]}
\def\Hz{\; \left[\mathrm{Hz}\right]}
\def\kg{\; \left[\mathrm{kg}\right]}
\def\kgpm{\; \left[\mathrm{\frac{kg}{\; m^3}}\right]}

\lhead{BMEGEMMBXVE, VEM alapjai 2.HF}
\cfoot{\thepage\ / \pageref{LastPage}}
\rhead{Németh Áron Imre, D1J5ZG}

\begin{python}
    import sympy as sp
    from sympy.printing.latex import LatexPrinter, print_latex
    from sympy import latex
    from calc.vem_hf2_calc import calculate
    from calc.var_print import printer, print_matrix, prin_TeX, my_latex

    V=calculate()
\end{python}

\begin{document}

\section{A szerkezet méretarányos ábrájának elkészítése}
\subsubsection*{Adatok:}

\begin{multicols}{2}
    \begin{itemize}
        \item $a=\pyc{prin_TeX(V["a"],1)}\m$
        \item $b=\pyc{prin_TeX(V["b"])}\m$
        \item $d=\pyc{prin_TeX(V["d_mm"])}\mili$
    \end{itemize}
    \columnbreak
    \begin{itemize}
        \item $E=\pyc{prin_TeX(V["E_GPa"])} \Gpa$
        \item $m_0=\pyc{prin_TeX(V["m_0"])} \kg$
        \item $\rho=\pyc{prin_TeX(V["rho"])} \kgpm $
    \end{itemize}
\end{multicols}

\subsection{A szerkezet léptékhelyes ábrája}
A szerkezet méretarányos ábrája esetünkben az alábbi módon néz ki:

\subsection{A szerkezet végeselem modelljének ábrája}
Az alábbi ábrán látható a szerkezet VEM modellje:

\section{Két BEAM elem használata}

\subsection{Csomópontok, elemek definiálása}
\subsubsection{Csomópontok definiálása}
Az alábbi táblázatban láthatóak a különböző csomópontok $x$ illetve
$y$ koordinátái:
\begin{center}
    \begin{tabular}{|c|c|c|}
        \hline
        Csomópont & x koordináta & y koordináta \\
        \hline
        \hline
        1         & 0            & $0$          \\
        \hline
        2         & $b$          & $0$          \\
        \hline
        3         & $a+b$        & $0$          \\
        \hline
    \end{tabular}
\end{center}

\subsubsection{Elemek hozzárendelése a csomó pontokhoz}
Az alábbi táblázat tartalmazza a csomópont-elem hozzárendeléseket:
\begin{center}
    \begin{tabular}{|c|c|c|}
        \hline
        Elemszám & Lokális 1. csomópont & Lokális 2. csomópont \\
        \hline
        \hline
        1        & 1                    & 2                    \\
        \hline
        2        & 2                    & 3                    \\
        \hline
    \end{tabular}
\end{center}

\end{document}